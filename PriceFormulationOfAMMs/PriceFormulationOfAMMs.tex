% !TEX TS-program = pdflatex
% !TEX encoding = UTF-8 Unicode

% This is a simple template for a LaTeX document using the "article" class.
% See "book", "report", "letter" for other types of document.

\documentclass[11pt]{article} % use larger type; default would be 10pt

\usepackage[utf8]{inputenc} % set input encoding (not needed with XeLaTeX)

%%% Examples of Article customizations
% These packages are optional, depending whether you want the features they provide.
% See the LaTeX Companion or other references for full information.

%%% PAGE DIMENSIONS
\usepackage{geometry} % to change the page dimensions
\geometry{a4paper} % or letterpaper (US) or a5paper or....
% \geometry{margin=2in} % for example, change the margins to 2 inches all round
% \geometry{landscape} % set up the page for landscape
%   read geometry.pdf for detailed page layout information

\usepackage{graphicx} % support the \includegraphics command and options

% \usepackage[parfill]{parskip} % Activate to begin paragraphs with an empty line rather than an indent

%%% PACKAGES
\usepackage{booktabs} % for much better looking tables
\usepackage{array} % for better arrays (eg matrices) in maths
\usepackage{paralist} % very flexible & customisable lists (eg. enumerate/itemize, etc.)
\usepackage{verbatim} % adds environment for commenting out blocks of text & for better verbatim
\usepackage{subfig} % make it possible to include more than one captioned figure/table in a single float
\usepackage{authblk}
\usepackage{amssymb}
\usepackage{amsthm}
\usepackage{amsmath}
\newtheorem{lemma}{Lemma}
% These packages are all incorporated in the memoir class to one degree or another...

%%% HEADERS & FOOTERS
\usepackage{fancyhdr} % This should be set AFTER setting up the page geometry
\pagestyle{fancy} % options: empty , plain , fancy
\renewcommand{\headrulewidth}{0pt} % customise the layout...
\lhead{}\chead{}\rhead{}
\lfoot{}\cfoot{\thepage}\rfoot{}

%%% SECTION TITLE APPEARANCE
\usepackage{sectsty}
\allsectionsfont{\sffamily\mdseries\upshape} % (See the fntguide.pdf for font help)
% (This matches ConTeXt defaults)

%%% ToC (table of contents) APPEARANCE
\usepackage[nottoc,notlof,notlot]{tocbibind} % Put the bibliography in the ToC
\usepackage[titles,subfigure]{tocloft} % Alter the style of the Table of Contents
\renewcommand{\cftsecfont}{\rmfamily\mdseries\upshape}
\renewcommand{\cftsecpagefont}{\rmfamily\mdseries\upshape} % No bold!

%%% END Article customizations

%%% The "real" document content comes below...

\title{Price Formulation of Constant Function Market Makers}
\author{Colin Grove}
\affil{HydraDX}
\date{} % Activate to display a given date or no date (if empty),
         % otherwise the current date is printed 

\begin{document}
\maketitle

\section{Price Functions}

\subsection{Definition of Price Functions}

The purpose of this research note is to explore a way of defining Constant Function Market Makers (CFMMs) by their price functions $p(x,y)$, where $x$ is the quantity of one asset, $y$ the quantity of the second asset, and $p(x,y)$ the spot price of $x$ with numeraire $y$.

Given intervals $I_1\subset (0,\infty)$ and $I_2\subset (0,\infty)$, we define $p:I_1 \times I_2 \to [0,\infty)$ to be a \textbf{price function} provided that $p(x,y)$
\begin{enumerate}
\item is non-increasing in $x$,
\item is non-decreasing in $y$,
\item is non-negative,
\item is continuous,
\item is Lipschitz continuous in $y$ on any closed interval $I \subset I_2$.
\end{enumerate}

We next prove existence of a CFMM with spot price equal to the any given price function.

\subsection{Existence of CFMM for $p(x,y)$}

We would like to solve the following ODE:
\begin{align*}
u'(x) &= -p(x,u(x))\\
u(x_0) &= y_0
\end{align*}

It follows from the last condition that the Picard-Lindelof theorem gives us a unique soluction $u(x)$ to the ODE on any closed interval $I\subset I_2$.
Since the solution is unique, it can be extended to all of $I_2$.

Observe then that $K(x,y) = \frac{y}{u(x)} = 1$ is a constant function from which a CFMM can be built.

Furthermore,
\begin{align*}
\frac{\partial K}{\partial x} &= -\frac{y}{u^2(x)}u'(x) = \frac{y}{u^2(x)}p(x,y)\\
\frac{\partial K}{\partial y} &= \frac{1}{u(x)}
\end{align*}

By the chain rule, the spot price is
$$
\frac{dy}{dx} = -\frac{\frac{\partial K}{\partial x}}{\frac{\partial K}{\partial y}} = -\frac{y}{u(x)}p(x,y)
$$

Since $y=u(x)$, the spot price of the CFMM using $K(x,y)$ as the constant function will be $p(x,y)$.

%\subsection{Relation to payoff function}



%\subsection{From a CFMM to a pricing function}

%Let $K:[0,\infty)^2 \to (0,\infty)$ be a differentiable, one-to-one function.


\subsection{Properties}

We define the weights intuitively as the percentages of the pool made up of each asset:
\begin{align*}
W_x &= \frac{xp(x,y)}{xp(x,y) + y}\\
W_y &= \frac{y}{xp(x,y) + y}\\
\end{align*}
Observe that this implies that $p(x,y) = \frac{W_x}{W_y}\frac{y}{x}$, the familiar constant product CFMM price formula.

\section{Example: Reweighting CFMM}

\subsection{Reweighting CFMM Definition}

We consider as an example price functions of the general form
$$
p(x,y) = C\left(\frac{y + \alpha}{x + \beta}\right)^{a+1},
$$
where $C,\alpha,\beta > 0$, and $a \geq -1$. Clearly $a=0$ is the constant product AMM and $a=-1$ is the constant sum AMM.

It turns out that $a>0$ gives us a family of reweighting AMMs.
Our ODE turns into
\begin{align*}
u'(x) &= - C\left(\frac{u + \alpha}{x + \beta}\right)^{a+1}\\
u(x_0) &= y_0
\end{align*}
This is separable, so we must simply solve
$$
\int (u + \alpha)^{-a-1} du = -C\int (x + \beta)^{-a-1} dx
$$
Doing this, we find the following swap invariant function:
$$
K(x,y) = \left((y + \alpha)^{-a} + C(x + \beta)^{-a}\right)^{-\frac{1}{a}}
$$

Note that
$$
W_x = \frac{xp(x,y)}{xp(x,y) + y} = \frac{xC\left(y + \alpha\right)^{a+1}}{xC\left(y + \alpha\right)^{a+1} + y(x + \beta)^{a+1}}
$$
When $\alpha = \beta = 0$, we see
$$
W_x = \frac{Cy^a}{Cy^a + x^a}
$$
It's clear from this equation that at $a=0$ the weight is constant (since it's just a constant product AMM), but with $a>0$, the AMM \emph{reweights} towards the token being purchased
(that is, if $x$ decreases and $y$ increases, $W_x$ increases).
The curvature of the reweighting CFMM is higher than that of the constant product CFMM, resulting in this reweighting.
In the language of AMMs, higher $a$ produces lower impermanent loss at the expense of subjecting traders to increased slippage.

\subsection{Asymptotes of the Reweighting CFMM}

We introduced the Reweighting AMM with $\alpha$ and $\beta$ not just for the sake of generalization, but because the choice of $\alpha = \beta = 0$ is problematic.
We would like $\lim_{x\to \infty} y = 0$ and $\lim_{y\to \infty} x = 0$, but we see that this requires
\begin{align*}
\alpha &= K(x,y)\\
\beta &= C^\frac{1}{a} K(x,y)
\end{align*}
We therefore adjust to
$$
K^{-a}(x,y) = (y + K(x,y))^{-a} + (C^{-\frac{1}{a}}x + K(x,y))^{-a}
$$
recalling that $K(x,y)$ is constant during a swap.

Note that this transition has actually adjusted our price function to
$$
p(x,y) = C\left(\frac{y + K(x,y)}{x + C^\frac{1}{a} K(x,y)}\right)^{a+1},
$$

\end{document}
